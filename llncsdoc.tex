% This is LLNCS.DOC the documentation file of
% the LaTeX2e class from Springer-Verlag
% for Lecture Notes in Computer Science, version 2.4
\documentclass{llncs}
\usepackage{llncsdoc}
\begin{document}
\markboth{\$ Simulator for Distributed Sleepy 
Consensus}{\$ Simulator for Distributed Sleepy Consensus}
\thispagestyle{empty}
\begin{flushleft}
\LARGE\bfseries Distributed Consensus\\[2cm]
\end{flushleft}
\rule{\textwidth}{1pt}
\vspace{2pt}
\begin{flushright}
\Huge
\begin{tabular}{@{}l}
Simulator for Distributed\\
Sleepy Consensus Protocol\\[6pt]
{\Large SJTU 2017 Cornell Summer Workshop}
\end{tabular}
\end{flushright}
\rule{\textwidth}{1pt}
\vfill
\begin{flushleft}
\large\itshape
\begin{tabular}{@{}l}
{\Large\upshape\bfseries Instructor }\\[8pt]
Elaine\enspace Shi \\[5pt]
Cornell\enspace University\\[5pt]
\end{tabular}
\end{flushleft}
\newpage
%
\section*{Our Group:}
%
\begin{flushleft}
\begin{tabular}{l@{\quad}l@{\hspace{3mm}}l@{\qquad}l}
$\bullet$&\multicolumn{3}{@{}l}{\bfseries Framework Team Members}\\[1mm]
% &\multicolumn{3}{@{}l}{Springer-Verlag}\\
% &\multicolumn{3}{@{}l}{Computer Science Editorial}\\
% &\multicolumn{3}{@{}l}{Tiergartenstra�e 17}\\
% &\multicolumn{3}{@{}l}{69121 Heidelberg}\\
% &\multicolumn{3}{@{}l}{Germany}\\[0.5mm]
 & Junxiang Huang       & & 841450297@qq.com\\
 & Yifei Pu       & &pkq2006@gmail.com\\[2mm]
$\bullet$&\multicolumn{3}{@{}l}{\bfseries Honest Node Team Members}\\[1mm]
 & Tiancheng Xie       & & wjxtcsgx@hotmail.com\\
 & Jiaheng Zhang      & &ZHANGJIAHENG@sjtu.edu.cn\\
 & Xiaotian You    &  & youxiaotian@hotmail.com\\
 & Shuyang Tang         &  & tangshuyang25@163.com\\
 & Chengyao Li   &  & cyli2014@sjtu.edu.cn\\[2mm]
 $\bullet$&\multicolumn{3}{@{}l}{\bfseries Adversary Members}\\[1mm]
 & Qingrong Chen      & & chenqingrong@sjtu.edu.cn\\
 & Ruisheng Cao      & &211314@sjtu.edu.cn\\
 & Shiquan Zhang    &  & zsq007@sjtu.edu.cn\\
 & Haochen Huang  &  & hhc98598@189.cn\\[2mm]
$\bullet$&\multicolumn{3}{@{}l}{\bfseries Integrators}\\[1mm]
 & Feiyang Qiu   &  & st.yeah@gmail.com\\
 & Lingkun Kong  &  & klk316980786@sjtu.edu.cn\\
 & Lanqing Liu    &  & sunnysunny@sjtu.edu.cn\\
 & Jialu Li          &  & 790359064@qq.com\\[2mm]
\\
\\
\\
\\
\\
\\
\\
% \noalign{\rule{\textwidth}{1pt}}
% \noalign{\vskip2mm}
$\bullet$&\multicolumn{3}{@{}l}{\bfseries Where to find our project?}\\[1mm]
 & https://github.com/initc3/sleepysim & & SleepySim\\[2mm]
\end{tabular}
\end{flushleft}


%
\newpage
\tableofcontents
\newpage
%
\section{Introduction}

\quad Consensus protocols are at the core of distributed computing and also provide a foundational building protocol for multi-party cryptographic protocols. In the paper about \emph{sleepy consensus} protocol \cite{Sleepy}, Rafael Pass and Elaine Shi propose a consensus protocol for realizing a “linearly ordered log” abstraction -- often referred to as state machine replication or linearizability in the distributed systems literature. They name it as \emph{sleepy consensus} protocol, which respects two important resiliency properties, i.e., consistency and liveness. And in \emph{sleepy consensus} model, players can be either online (alert) or offline (asleep), and their online status may change at any point during the protocol.

In this paper, we build a simulator for monitoring the real-world performance of \emph{sleepy consensus} protocol by constructing a framework which implements \emph{sleepy consensus} protocol, as well as imitating behaviors of honest players and corrupted/adversarial players in the meanwhile. After analyzing the simulating results, \textbf{we know ... add text here}

This document is organized as follows. In Section 2, we introduce the framework of simulator. In Section 3, we present how honest players work while simulating. And in Section 4, we imitate the adversarial players' behavior and attack the \emph{sleepy consensus} protocol by several algorithms. We give the analysis of simulating results in Section 5. Finally, we draw conclusions in Section 6.

\section{The Framework of Simulator}
%
In this section, we will illustrate the construction of the framework of our simulator, which includes \textbf{add text here}

%
\subsection{Controller}
%
The LLNCS class is an extension of the standard \LaTeX{} ``article''
document class. Therefore you may use all ``article'' commands for the
body of your contribution to prepare your manuscript.
LLNCS class is invoked by replacing ``article'' by ``llncs'' in the
first line of your document:
\begin{verbatim}
\documentclass{llncs}
%
\begin{document}
  <Your contribution>
\end{document}
\end{verbatim}
%
\subsection{Contributions Already Coded with \protect\LaTeX{} without
the LLNCS document class}
%
If your file is already coded with \LaTeX{} you can easily
adapt it a posteriori to the LLNCS document class.

Please refrain from using any \LaTeX{} or \TeX{} commands
that affect the layout or formatting of your document (i.e. commands
like \verb|\textheight|, \verb|\vspace|, \verb|\headsep| etc.).
There may nevertheless be exceptional occasions on which to
use some of them.

The LLNCS document class has been carefully designed to produce the
right layout from your \LaTeX{} input. If there is anything specific you
would like to do and for which the style file does not provide a
command, {\em please contact us}. Same holds for any error and bug you
discover (there is however no reward for this -- sorry).
%
\section{The Imitation of Honest Players}
%
With mathematical formulas you may proceed as described
in Sect.\,3.3 of the {\em \LaTeX{} User's Guide \& Reference
Manual\/} by Leslie Lamport (2nd~ed. 1994), Addison-Wesley Publishing
Company, Inc.

Equations are automatically numbered sequentially throughout your
contribution using arabic numerals in parentheses on the right-hand
side.

When you are working in math mode everything is typeset in italics.
Sometimes you need to insert non-mathematical elements (e.g.\
words or phrases). Such insertions should be coded in roman
(with \verb|\mbox|) as illustrated in the following example:
\begin{flushleft}
{\itshape Sample Input}
\end{flushleft}
\begin{verbatim}
\begin{equation}
  \left(\frac{a^{2} + b^{2}}{c^{3}} \right) = 1 \quad
  \mbox{ if } c\neq 0 \mbox{ and if } a,b,c\in \bbbr \enspace .
\end{equation}
\end{verbatim}
{\itshape Sample Output}
\begin{equation}
  \left(\frac{a^{2} + b^{2}}{c^{3}} \right) = 1 \quad
  \mbox{ if } c\neq 0 \mbox{ and if } a,b,c\in \bbbr \enspace .
\end{equation}

If you wish to start a new paragraph immediately after a displayed
equation, insert a blank line so as to produce the required
indentation. If there is no new paragraph either do not insert
a blank line or code \verb|\noindent| immediately before
continuing the text.

Please punctuate a displayed equation in the same way as other
ordinary text but with an \verb|\enspace| before end punctuation.

Note that the sizes of the parentheses or other delimiter
symbols used in equations should ideally match the height of the
formulas being enclosed. This is automatically taken care of by
the following \LaTeX{} commands:\\[2mm]
\verb|\left(| or \verb|\left[| and
\verb|\right)| or \verb|\right]|.
%
\subsection{lalala}
%
\begin{alpherate}
\item
In math mode \LaTeX{} treats all letters as though they
were mathematical or physical variables, hence they are typeset as
characters of their own in
italics. However, for certain components of formulas, like short texts,
this would be incorrect and therefore coding in roman is required.
Roman should also be used for
subscripts and superscripts {\em in formulas\/} where these are
merely labels and not in themselves variables,
e.g. $T_{\mathrm{eff}}$ \emph{not} $T_{eff}$,
$T_{\mathrm K}$ \emph{not} $T_K$ (K = Kelvin),
$m_{\mathrm e}$ \emph{not} $m_e$ (e = electron).
However, do not code for roman
if the sub/superscripts represent variables,
e.g.\ $\sum_{i=1}^{n} a_{i}$.
\item
Please ensure that {\em physical units\/} (e.g.\ pc, erg s$^{-1}$
K, cm$^{-3}$, W m$^{-2}$ Hz$^{-1}$, m kg s$^{-2}$ A$^{-2}$) and
{\em abbreviations\/} such as Ord, Var, GL, SL, sgn, const.\
are always set in roman type. To ensure
this use the \verb|\mathrm| command: \verb|\mathrm{Hz}|.
On p.\ 44 of the {\em \LaTeX{} User's Guide \& Reference
Manual\/} by Leslie Lamport you will find the names of
common mathe\-matical functions, such as log, sin, exp, max and sup.
These should be coded as \verb|\log|,
\verb|\sin|, \verb|\exp|, \verb|\max|, \verb|\sup|
and will appear in roman automatically.
\item
Chemical symbols and formulas should be coded for roman,
e.g.\ Fe not $Fe$, H$_2$O not {\em H$_2$O}.
\item
Familiar foreign words and phrases, e.g.\ et al.,
a priori, in situ, brems\-strah\-lung, eigenvalues should not be
italicized.
\end{alpherate}
%
\section{The Imitation of Honest Players}
\label{refedit}
%
\subsection{Headings}\label{headings}
%
All words in headings should be capitalized except for conjunctions,
prepositions (e.g.\ on, of, by, and, or, but, from, with, without,
under) and definite and indefinite articles (the, a, an) unless they
appear at the beginning. Formula letters must be typeset as in the text.
%
\subsection{Capitalization and Non-capitalization}
%
\begin{alpherate}
\item
The following should always be capitalized:
\begin{itemize}
\item
Headings (see preceding Sect.\,\ref{headings})
\item
Abbreviations and expressions
in the text such as  Fig(s)., Table(s), Sect(s)., Chap(s).,
Theorem, Corollary, Definition etc. when used with numbers, e.g.\
Fig.\,3, Table\,1, Theorem 2.
\end{itemize}
Please follow the special rules in Sect.\,\ref{abbrev} for referring to
equations.
\item
The following should {\em not\/} be capitalized:
\begin{itemize}
\item
The words figure(s), table(s), equation(s), theorem(s) in the text when
used without an accompanying number.
\item
Figure legends and table captions except for names and abbreviations.
\end{itemize}
\end{alpherate}
%
\subsection{Abbreviation of Words}\label{abbrev}
%
\begin{alpherate}
\item
The following {\em should} be abbreviated when they appear in running
text {\em unless\/} they come at the beginning of a sentence: Chap.,
Sect., Fig.; e.g.\ The results are depicted in Fig.\,5. Figure 9 reveals
that \dots .\\
{\em Please note\/}: Equations should usually be referred to solely by
their number in parentheses: e.g.\ (14). However, when the reference
comes at the beginning of a sentence, the unabbreviated word
``Equation'' should be used: e.g.\ Equation (14) is very important.
However, (15) makes it clear that \dots .
\item
If abbreviations of names or concepts are used
throughout the text, they should be defined at first occurrence,
e.g.\ Plurisubharmonic (PSH) Functions, Strong Optimization (SOPT)
Problem.
\end{alpherate}
%
\section{The Analysis of Simulating Results}
Here is the analysis of Simulating Results.
\section{Conclusion}
In conclusion....


\begin{thebibliography}{}  % (do not forget {})

\bibitem{Sleepy}
Pass, Rafael, and Elaine Shi. \emph{The sleepy model of consensus}. Cryptology ePrint Archive, Report 2016/918, 2016. http://eprint. iacr. org/2016/918, 2016.

\end{thebibliography}
%
\end{document}
